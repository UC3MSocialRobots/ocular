\chapter {Regulatory compliance}

The present section covers the regulatory compliance that affects directly to the system presented. 
% It must be noted that since the project is a research project, most of the regulations here exposed may not be needed. 
% Nevertheless, they must be taken into account in the case that the system is commercialized.

	% \paragraph{ISO}\mbox{}\\

	% The International Organization for Standardization is an international organism that settles international standards. 
	% It has several joint committees with the International Electrotechnical Commission (IEC). 
	% They develop standards in different technical fields such as the electrical, electronic or IT fields. 
	% \\

	% The Information Technology (IT) term is related to the use of computers and telecommunications to interchange, store and manipulate data.  
	% It is a broad field in which the present project may be included.
	% The Joint Technical Committee devoted to this subject is the ISO/IEC JTC 1. 
	% Its main mission is develop, maintain, promote and facilitate IT standards regarding different areas such as the following : 
	% \begin{itemize}
	% 	\item{Design and development of systems and tools}
	% 	\item{Define performance and quality standards}
	% 	\item{Security, interoperability and portability of IT systems and products}
	% 	\item{Unified tools and environments as well as harmonized IT vocabulary}
	% \end{itemize}

	% This Committee has recently included the regulations regarding assistive tecnology (AT) \cite{japan1}\cite{japan2}.
	% %--> http://en.wikipedia.org/wiki/ISO/IEC_JTC_1 [IT projects] 


	% \paragraph{Software regulations}\mbox{}\\
	
	There are different regulations regarding software. 
	% This section is centered on the ones related to Open Source systems, since the software developed in this thesis is Open  
	Since the project is Open Source, the section is centered on presenting the regulations of this type of systems. 
	Nowadays, the author of the software has the right of sharing it using a contract. 
	In it he determines which of the author rights he is going to yield and under what conditions. 
	This type of contract is called a software license. 


	\section{OCULAR}

	This project involves taking personal information to construct a dataset with which recognize different objects. 
	In order to observe the data protection Spanish law (Ley Orgánica 15/1999, de 13 de diciembre, de Protección de Datos de Carácter Personal), the stored data does not contain any personal data. 
	Instead of storing an image, the information retrieved is a matrix of numbers containing the descriptors extracted from the objects. 
	This ensures that the system could be used in commercial software or that further investigations could be performed accordingly with the law. 

	This project is being distributed with a MIT License (MIT). 
	This license may be found in this  \href{https://raw.githubusercontent.com/irenesanznieto/ocular/master/LICENSE.md}{\color{blue}\underline {link}}, and states the following: \\

	"Copyright (c) 2014 Irene Sanz Nieto

Permission is hereby granted, free of charge, to any person obtaining a copy of this software and associated documentation files (the "Software"), to deal in the Software without restriction, including without limitation the rights to use, copy, modify, merge, publish, distribute, sublicense, and/or sell copies of the Software, and to permit persons to whom the Software is furnished to do so, subject to the following conditions:

The above copyright notice and this permission notice shall be included in all copies or substantial portions of the Software.

The software is provided "as is", without warranty of any kind, express or implied, including but not limited to the warranties of merchantability, fitness for a particular purpose and noninfringement. In no event shall the authors or copyright holders be liable for any claim, damages or other liability, whether in an action of contract, tort or otherwise, arising from, out of or in connection with the software or the use or other dealings in the software."
\\

	The last paragraph is published in uppercase letters, but it was converted to lowercase to avoid a disturbance in the structure of the thesis. 	
	The license claims that software is available for redistribution and use, but no warranty is provided with it. 
	The different algorithms and third-party packages were selected taking into account that their licenses must be compatible with the one being provided by this system. 
	In the next sections the licenses under which the different packages are distributed are presented. 


	\section{ROS}
	All ROS core code is distributed under a BSD license, more specifically a BSD 3-Clause license. 
	It is very similar to the OCULAR license, the redistribution is permitted under certain conditions. 
	More information may be found in this \href{http://opensource.org/licenses/BSD-3-Clause}{\color{blue} {webpage}}. 
	The different ROS packages that are used in this thesis (openni\_camera, openni\_launch and pi\_tracker) are distributed under a BSD license, according to their web pages ( \href{http://wiki.ros.org/openni_camera}{openni\_camera}, \href{http://wiki.ros.org/openni_launch}{openni\_launch}, \href{http://wiki.ros.org/pi_tracker}{pi\_tracker}).

	\section{OpenCV}
	The Open Source Computer Vision library is released as well under a BSD license. 
	Hence it is free for both academic and commercial use. 
	Nevertheless, there are certain algorithms implemented whose license is different from the whole library. 
	The SIFT or SURF descriptors are two examples of this fact. 
	These algorithms have a software patent. 
	This legal figure allows the use of the algorithms for investigation purposes. 
	But for commercial uses the payment of a fee is imposed. 
	This was one of the reasons of choosing the ORB algorithm as the descriptor extractor of the system. 
	ORB has no patent and hence could be used for both commercial and research. 

	\section{PCL}
	The Point Cloud Library has as well a BSD license. 
	It is then free for commercial and research use. 
	Further information about the library and its license may be found in this \href{http://pointclouds.org}{\color{blue} {webpage}}. 


	% \paragraph{Data protection} \mbox{}
	\\

	% Since this project implements a proof of concept, the experiments performed with it were done by the author. 
	

	% This law "warrants and protects the personal data treatment, the public liberties and the fundamental rights of the persons, specially their honor and personal and familiar intimacy".
	% It is applicable to all data stored in a physical support. 
% 	Privacidad y confidencialidad: cualquier investigación que contenga datos de caracter personal tiene que cumplir los preceptos de la legislación de protección de datos. En España la norma que regula estos aspectos es la Ley Orgánica 15/1999, de 13 de diciembre, de Protección de Datos de Carácter Personal, cuayo objeto es 'garantizar y proteger en lo que concierne al tratamiento de los datos personales, las libertades públicas y los derechos fundamentales de las personas físicas, y especialmente de su honor e intimidad personal y familiar”.
% La ley es de aplicación a los datos de carácter personal registrados en cualquier soporte físico. El tratamiento de los datos cubre las actividades de recolección, registro, almacenamiento, recuperación, consulta, uso y diseminación. Para garantizar el derecho a la protección de datos, es necesario informar a las personas implicadas y solicitar su consentimiento para el tratamiento de sus datos. 