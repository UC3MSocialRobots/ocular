\begin{abstract}

% \large{\textbf{OCULAR: In-hand object detection and tracking using 2D and 3D information}}
% \\

% Irene Sanz Nieto, July 11 2014.


\chapter{Abstract}
% This bachelor's thesis consists in a software that implements an in-hand object learning and recognition using 2D and 3D information. 
% The code is Open-Source and it was designed to be running inside a robot. 
% The idea is to create a new stand-alone module that could be included in different robots. 
% This package would aid in the environment perception of the robot. 
% \\

% The thesis is structured as follows. 
% First, the purpose and motivations are presented and a brief description of the project is performed. 
% Afterwards, the state of the art is explained and the innovations performed in this thesis with respect to the previous art are highlighted. 
% Then, the system developed is described as well as the experiments performed on it. 
% Finally, the results are shown and discussed and the conclusions derived from them explained.
% This project is aimed as a proof of concept study and hence the possible improvements observed from the experiments are also presented in the last section. 
% \\

% The analysis of the system demonstrated an increase of the accuracy with the increment of views per object acquired.
% The results depicted the robustness of the system as well, since it is able to differentiate between similar objects a high number of times  



This thesis presents a system that is able to learn new objects that are being held by a human, and track previously learned objects. 
This information can be useful for the robot's interaction with its environment. 
In particular, the introduction of robots to assist humans could improve our quality of life. 
% The first world's population is ageing. 
% This fact is going to affect the economy of these countries since elder people need more care and are not able to work. 
% The introduction of robots to assist in the care of elder people could help in this situation. 
Nevertheless, nowadays most robots are able to interact with its environment in a limited way. 
\\

In order to perform complicated tasks such as human aiding, the robot's perception systems should be improved. 
The recognition of the tasks the human is performing is important to understand the situations. 
The objects the person is holding gives a huge amount of information about the task being performed, i.e. if he is holding a book, he is probably going to read. 
Having a system that could identify the objects that are being held could improve the situation recognition of the robot. 
\\

This thesis presents a system that is able to track the user's hand and recognize the objects the user is holding. 
The software is able to easily learn new objects. 
It is possible to obtain a defined number of views per object to improve the later recognition. 
The system also incorporates an easy and intuitive pose interface to interact with it. 
The results of the experiments performed on the system can be seen in section \ref{results}. 
They show that the system improves its accuracy with the number of views per object. 
Also, the confusion matrices reveal a good robustness of the algorithm when using similar objects. 
The software is able to learn and recognize in real time, which is very useful to obtain a good interaction with the robot. 
\\

The system is intended to be a piece in the situation recognition algorithm of the robot. 
It is able to effectively learn and recognize objects that are hand-held, giving information that can be key to determine the tasks the human is developing. 
It could be a piece of information useful to develop a better human - robot interaction for social and assistive robots. 
It may, in fact, be introduced in the robot that probably will take care of us when we are older. 


\end{abstract}


\chapter{Resumen}
\begin{abstract}

\end{abstract}