%\section{Introduction}
\section{History}

The evolution of computer vision might be seen as the evolution of the optimism of the investigators with respect to this field. 
In the 1950s the first computers were being developed. There were many science-fiction films and books that suggested a new robotic era approaching. The optimism and the confidence in the power of computers were enormous. 
Since seeing is easy for humans and the computers had a big capacity for computing, it was thought that the artificial vision field (as it was called in that period) was going to grow rapidly. 
\\

In the 1960s the investigators started to frustrate. 
% They realized that, even though the hardware used for computer vision was better than the one the humans possess, the processing made by the mind was much better and complex than the algorithms developed at the time. 
They realized the complexity of the data processing made by the mind. 
The fact that the information received is a two-dimensional projection of a three-dimensional world requires that the processing step interpret and make an abstraction of that data.
Most of the human mind was still a mystery, and since it was not known how it processes the information received by the eyes it was not possible to reproduce it in a software. 
At this time, it was discovered the importance of the learning stage in the recognition of objects. 
Larry Roberts is accepted as the father of the Computer Vision field\cite{Huang}.
His Ph.D. thesis presented on June 1963\cite{Roberts1963} was the model that later scientists followed in their investigations. 
It studies the possibility of making a computer construct and display a three-dimensional array of objects given a single two-dimensional picture. 
\\

% The frustration and little advance in the field lasted two decades. 
 It was in the 1980s when the investigators changed the way of thinking about the field.
  % They changed the name to the one currently being used: "Computer Vision" and also changed the main investigation branches.
They realized that the research must be centered in the low-level vision to process the input images and make them more easy to interpret and manage to the computer \cite{Huang}. 
The evolution of the hardware (both computers and specialized hardware) also influenced the development of more complex projects. 
\\

\section{Computer vision today}

Nowadays the computer vision field is divided in many different branches. 
Some of these branches are object recognition, learning, indexing, video tracking, scene reconstruction, event detection, motion estimation and image restoration, among others. 
Nevertheless, they are not separate fields of study. 
Often the systems perform tasks that belong to different branches in order to achieve their goal. 
% It is noticeable that inside each branch there are multiple variations in the systems developed. 
% They vary depending on the application that motivated the creation of the system. 
% This creates a huge amount of different systems that has different input device or even different input conditions for example the location of the user in the image. 
% The definition of the previous image processing prior object learning and recognition changes between the different studies due to this fact.
% \\

In this thesis I present a system that allows to easily learn and recognize hand-held objects in real time. 
It could be classified inside the object recognition and learning, video tracking and event detection branches of study. 
This chapter is devoted to the discussion of the previous literature related to this topic. 
First the most similar systems to this thesis are presented and discussed. 
Afterwards the common items to these systems are presented as well as the most suitable alternatives for the application presented here. 


