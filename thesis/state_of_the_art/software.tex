\section{Object learning and recognition}

Object learning and recognition is an active field. 
In this section the most important works related to that field are presented. 
Also, the differences and improvements performed in this thesis with respect to the previous art are estated. 
% In this section the most important works related to this thesis are exposed. 

\subsection{Object learning and recognition using 2D and 3D input data}

This special branch in the object learning and recognition field has been active for many years. 
One of the first works in the field was the one performed by D.M.Gavrila and F.C.A. Groen. 
In 1992,  they developed a  system that performs 3D object recognition using 2D input data \cite{D.M.Gavrila1991}.
A 3D model was created and then the new 2D projection was matched to this model.
A geometric hashing method of matching was applied to the input 2D projection. 
The image was only matched to a model if the error to a nearby 2D projection was within certain thresholds.
The testing was performed under a controlled environment and using textured objects. 
The system would not give the same results in a real environment with noise and loosely defined objects. 
Also, the use of hashing limits the size of the dataset due to its high computing cost. 
\\

Later systems as \cite{Sheta2012} explore different descriptors for the objects that result less time-consuming. 
In this case, fuzzy logic is used to match the learned features to the new ones. 
Those descriptors are a set that include Affine, Zemike and Hu moments invariants among others. 
The results are promising, obtaining a high percentage of true positives. 
But, again, the system uses a controlled setup:
The input is collected from three cameras that has a known illumination and orientation. 
The objects are white with significantly different shapes and they rest in a black background during learning and recognition phases. 
The applications in which it could be used are more related to the industrial field than the robotics field. 
\\

Outdoors, \cite{Zia2013} demonstrated the effectiveness of two methods when combined: local descriptors and 3D wireframes. 
The system was able to distinguish between cars and bicycles and to estimate their pose. 
The input of the recognition is a single 2D image. 
Nevertheless, the training was made off-line, using a high number of Computer-Aided Design (CAD) model's views. 

These systems differ with the one presented in this thesis in one major aspect: 
the user-software interaction. 
This thesis is based on providing an easy and intuitive interface. 
The learning process is performed easily and on-line and the recognition is real-time. 
There is no setting needed, no previous processing and care about the environment. 
The combination of 2D and 3D information in my system is, to the best of my knowledge, unique. 
The fact of recognizing the objects using those two different methods and then apply a decision algorithm better the results, reducing the effect of the noise.  



\subsection{In-hand object learning and recognition}

Most of the literature in object learning and recognition has been developed using wearable cameras as the input of the system. 
One of the most representative works is \cite{Roth2006}, which consists on a camera used to capture daily objects to construct a dataset more easily. 
Its main feature is that it eliminates the necessity of manually segmenting and labeling the learning datasets for object recognition libraries. 
It only learns new templates that are later feed to a off-line learning algorithm. 
Another example is \cite{Philipose2009}, in which a benchmarking of an egocentric object recognition system is performed. 
This in-hand object recognition uses as input an image that has the object used at the center. 
The background is not segmented in this system. 

\\

This thesis presents a different approach: an in-hand object recognition and learning in which the user is located in front of the acquisition device. 
It is a solution to the same problem, the recognition of an object that a user holds. 
But the initial setting is different. 
In my thesis, the user is in front of the camera, because the software is running on a robot. 
The interaction between the software and the user is different from the previous art on the field. 
This change in the input creates an additional difficulty due to the hand's segmentation in the input image. 
Furthermore, to the best of my knowledge there is no previous art in in-hand object learning and recognition using 2D and 3D combined together. 
All of the algorithms used a camera and hence 2D information as input. 
The inclusion of 3D and the independence between the 2D and 3D recognition processes improve the noise resistance of my system with respect to the previous art. 


% \paragraph{Template matching}


In the object recognition field obtaining good descriptors is of paramount importance. 
The quality of the features extracted as the model of the object determine the quality of the algorithm. 
Due to this importance, the next section is devoted to the descriptors that can be obtained from 2D and 3D data.

