\section{Objectives}

The objectives of this thesis are listed below. 

\begin{itemize}
	\item To develop a system capable of recognizing objects in real time. 
			By recognition it is meant the detection of new objects and the comparison with a previously obtained dataset outputting the ID number of the most similar template.
	\item To develop a system capable of learning objects in real time. 
			This includes the storage of the dataset for further usages.
			The learning process is performed acquiring a definable number of views per object. 
	\item The system must be able to detect the persons that are in front of the robot. 
	\item The software must detect the location of the hands of the user in order to extract from there the object to be recognized and learned. 
	\item The system must be able to learn more than one view of each object. 
	\item The software must have a gestural interface that allows to trigger the recognizing and the learning modes. 

\end{itemize}
%In order to achieve those objectives, an spiral software development model was applied. In this model, the software analysis, objectives definitions, prototypes creation and testing are done iteratively. 
\\

%The intermediate objectives of the thesis that were defined in each iteration are enumerated below. 

% \begin{itemize}

% 	\item{Develop or adapt an existing hand tracking software}
% 	\item{Develop a Region Of Interest extractor software in order to filter the input raw 2D information. }
% 	\item{Develop a Region Of Interest extractor software in order to filter the input raw 3D information. }
% 	\item{Develop an object learning software using 2D data that allows to extract multiple views from each object}
% 	\item{Develop an object learning software using 3D data that allows to extract multiple views from each object}
% 	\item{Develop an object recognition software using 2D data and multiple object's views}
% 	\item{Develop an object recognition software using 3D data and multiple object's views}
% 	\item{Develop a feedback system to reduce the recognition uncertainties}
% 	\item{Develop a gesture interface to control the software}
% 	\item{Compare the object recognition using 2D and 3D information, the effectiveness and efficiency of each algorithm}
% 	\item{Perform a comparative study when changing the number of views used per object}

% \end{itemize}
