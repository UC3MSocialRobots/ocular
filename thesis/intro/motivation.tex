\section{Motivation}
Technology has evolved enormously in the past years. The economic situation described above forced many experienced professionals to trying their luck creating start-ups. Many of the ideas of those enterprises are having nowadays a huge impact on the society. 
\\

One of these open-source projects that appeared was the low-cost 3D printers. These machines have changed the manner of investigating many fields, since they allow to design different pieces easily and have a 3D reproduction in a few hours. 
\\

Initially, they were used in investigation and more specifically in robotics, but now they are used in many different fields. Among those fields, there are medicine, construction or even food making. 
\\

In medicine, they have been a revolution since they allow to create customized and precise pieces in very few time. They have been used for prosthesis and implants for persons of various ages, even for babies. In the prosthesis fields in particular, the 3D printing technology is being a complete revolution. Before, the prosthesis were very expensive and permitted only fixed movements and combinations. The adaptations to each individual were made in the final product itself, trying to make it as comfortable as possible for the wearer. Nowadays, the prosthesis are customized for each patient, reducing the inconveniences and increasing their usability. Also, they can be easily and cheaply adapted for children as an example, who are still experimenting many changes in their bodies. They are much cheaper than they were before, and everyone with a 3D printer may construct one. 
In order to 3D print a piece a file with its description is needed. There are many web-pages that store open-source designs that ranges from decoration models to complex prosthesis. This fact is decisive because there is not needed a huge amount of knowledge or money to improve the life quality of a person using these technologies. 
\\

There are numerous open-source projects and developers that put in common their knowledge to improve the technology being used. I have used many of them in the previous years, to learn about 3D printing, robotics or programming among other fields. 
\\

It is a fact that acquiring knowledge would be much difficult if the Open Source initiative has not been invented. This impulsed me towards developing something useful and that could be used by other people. The idea of creating a software that could be used in robotics investigation but also help people at the same time. 
\\[0.5cm]

Many of the projects are aimed at aiding physically impaired people, creating sternal skeletons and robotic arms that could aid them. But a personal fact led me to realize that visually impaired people were not having as much attention. The applications developed for them are still rough to use and also it is difficult for a grown person to develop his remaining senses to supply the information lost. 
\\

Besides, in the robotic field new lines of investigation have appeared. The social robots are now a reality and in the near future we will interact with them everyday. In order to understand the human behavior, the recognition of the objects being handled by them is crucial. 
\\[0.5cm]


Computer vision has experience an important improvement in the last years through the upgrade of the hardware and software that compose it. The hardware such as acquisition elements (cameras, depth sensors, etc) and computing elements (PCs or other programmable devices) have experienced a rapid advance in the past years. It allowed to process more data that is now obtained more accurately and with less noise. This increase in the computing power of the equipment created a possibility of introducing more complex libraries and frameworks and even operating systems. 
\\

Now, the technology is available to solve the problems presented, the aid of visually impaired people and the introduction of new information in the social robotics field. This is how the idea behind this thesis appeared: the creation of a modular software that implements an in-hand object recognition algorithm. 
