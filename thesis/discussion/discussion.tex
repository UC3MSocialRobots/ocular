%\addcontentsline{toc}{part}{Discussion}
\chapter{Discussion}
\label{discussion}

This chapter covers the discussion of the tests results presented in the previous section, number \ref{results}.
It follows the same structure than the last two parts. 
First, the benchmarking of both the nodes and the topics is discussed. 
Afterwards, the results obtained in the accuracy experiment are explained and justified. 
\\

The last section of the thesis is devoted to the improvement of the system based on the observations performed on the experiments. 

\section{Performance testing}

	\begin{itemize}
		\item{\textbf{Package Benchmarking}}
		\\

			The nodes with a higher computing consumption are the ROI segmenters and the feature extractors both 2D and 3D. 
			Since the 3D data has a higher size, the usage of the nodes using 3D information is much higher than those processing 2D data. 
			\\

			The learner recognizer node also has a higher consumption than the converter, event handler or system output nodes. 
			This is because the last ones perform simple conversions and computations of integers and floating data. 
			On the other hand, the learner recognizer node implements the state machine of the software. 
			This means it has to deal with a higher amount of data than the previous nodes. 




		\item{\textbf{Topic Benchmarking}}\\

	\end{itemize}