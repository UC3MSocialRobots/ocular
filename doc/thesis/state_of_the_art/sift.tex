\addcontentsline{toc}{subsection}{SIFT}
\subsection*{SIFT}

SIFT (Scale Invariant Feature Transform) is a scale and rotation invariant feature descriptor\cite{sift}. 

There are various papers in which the performance of SIFT is compared with other descriptors, such as \cite{Mikolajczyk2005}. In them, it can be seen that SIFT outperforms the others, mainly due to its combination of local information and relative strengths and orientations of gradients. This combination makes it more robust to illumination and viewpoint changes and the addition of noise. 
\\

In order to minimize the cost of extracting such a distinctive features, a cascade filtering approach is used in order to apply the most time-consuming operations only at locations that pass an initial test. 
\\

Its relation between distinctiveness and speed is good. It can be used for on-line applications but it still has a latency that could be improved. Hence, it is an almost real-time algorithms. As an example, in \cite{sift_fpga} the SIFT algorithm was implemented on a FPGA (Field Programmable Gate Array), improving its speed by an order of magnitude and thus allowing it to run in real-time.
\\

The main reason of this high computing time, which is acceptable for on-line applications but improvable, is the descriptor vector size. In the aim of creating a highly distinctive descriptor, the vector is over-dimensioned slowing the detection, description and matching processes. 
\\

In relation with object recognition, this algorithm has a good performance in medium cluttered spaces. If the image is cluttered, there will appear a number of features of the background that do not have a match in the given database. Hence, it will give false positives and the match will have a lower probability. 

