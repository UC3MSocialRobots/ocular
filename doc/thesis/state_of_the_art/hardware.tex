\addcontentsline{toc}{chapter}{Hardware}
\chapter*{Hardware}

 
%%%%%%% INTRODUCTION %%%%%%
\addcontentsline{toc}{section}{Introduction}
\section*{Introduction}
The advance in the computer vision discipline is greatly linked with the development in the hardware. The main components of a computer vision system are the following: 

\begin{itemize}
	\item{Power Supply: } Device needed by the other components in order to work. 
	\item{Acquisition device: } Device that captures the world and represents it as an array of data. That data can be two or three dimensional. 
	\item{Processing unit:} Receives the information from the acquisition device and processes it. It is usually programmable. Nowadays the most used processing units are PCs. 
	\item{I/O unit: } Serves as a bridge between the acquisition device and the processing unit if needed. 

\end{itemize}

In this chapter the state of the art of the different acquisition devices is going to be presented. 

%%%%%% ACQUISITION DEVICES %%%%%%
\addcontentsline{toc}{section}{Acquisition devices}
\section*{Acquisition devices}
There are different acquisition devices being used in the computer vision field. They are usually classified depending on the output data they provide: 

\begin{itemize}
	\item{Cameras:}	The output data is two-dimensional. 
	\item{RGB-D sensors:} The output data is three-dimensional. 
\end{itemize}

The usage of one or another acquisition device depends on the application. The RGB-D sensors provide a higher number of information than the cameras. They reduce the ambiguities produced by the cameras when projecting the three-dimensional world into to dimensions. But also the RGB-D sensors output a higher amount of data. 
That is why, using three-dimensional information as the input of a software requires a higher-capacity processing unit than using two-dimensional data.



%%%%%% CAMERAS %%%%%%
\addcontentsline{toc}{section}{Cameras}
\section*{Cameras}




%%%%%% RGB-D SENSORS %%%%%%
\addcontentsline{toc}{section}{RGB-D Sensors}
\section*{RGB-D Sensors}
Computer vision is a field that needs specific hardware to retrieve a description of the world. This description has been done for a number of years in two dimensions. But this changed when the first version of an affordable RGB-D sensor appeared in 2010: the Kinect.
\\

This sensor was designed to be used in games, but developers soon realized the huge potential of the hardware for Computer Vision.  
Now, instead of a two-dimensional information as an input it was possible to have three-dimensional information. 
\\

Two-dimensional information is a projection of a three-dimensional world and so there appear ambiguities when using that data. That is why having three-dimensional information is a big advantage. 
\\


